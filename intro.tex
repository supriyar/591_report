Offloading mobile applications to the cloud helps to bridge the gap between the computing capabilities
of mobile devices and server machines. With the advance of ubiquitous computing many low-powered computing
devices can benefit from the high processing powers of available server machines. The elastic computational
cloud fuctionality helps in augmenting the processing speeds on demand and also leads to power efficient execution
of applications. A lot of research work ~\cite{netobj}, ~\cite{emerald}, ~\cite{tcl} has gone into designing systems where
objects can seamlessly migrate between different end-points to exploit better resources. These systems create
programming language constructs that help to facilitate offloading.

Mobile code offloading may involve analyzing the mobile application statically and partitioning it to identify the
most computationally expensive operations. These operations can be run on a remote server when the client is well
connected to the server. Theoretically, offloading is advisable in scenarios where the network conditions are good
for offloading. Moreover, the mobile application should preferably have high amounts of computational processing
which can execute much faster remotely. Therefore, a good offloading engine should take into account the amount of data that
needs to be transfered to the server and compare it with the current network conditions to make decisions.
This ensures that the offloading capabilities of the client do not affect its perfomance adversely. 
Our main goal in this project is to improve this offloading decision for an already existing offloading infrastructure.

The backbone for our offloading decision is the COMET (Code Offlaod by Migrating Execution Transparently)
infrastructure. COMET has built its offloading engine
on the Dalvik Virtual Machine. COMET uses Distributed Shared Memory (DSM) to implement offloading. It can be used to offload
any android application in general as the application need not contain any offloading logic. By using a DSM based approach, the
offloading mechanism can support multi-threaded computation and allow threads to move between the client and the server.
The current scheduling algorithm in COMET is based on a simple heuristic and can be improved. Moreover, COMET aborts the
execution at the server when a native method is encountered. In an application with plenty of native calls, continuous abortion
at the server might lead to performance loss as time is wasted in the data transfer and sync between client and server.
To address these limitations, we use static analysis to devise heuristics that determine when methods should be offloaded.
\newline
\newline
Our primary contributions through this project are the following
\begin{itemize}
   \item We perform static analysis of applications to identify native methods and their parent methods
   \item We compare the data transfer during sync operations to the current network bandwidth to make offloading decision
   \item Based on method execution times, we implement two offloading heuristics that identify the methods which can be marked
   as safe or unsafe for migration
\end{itemize}
\vspace{2mm}
The remainder of this report is organized as follows: Section 2 presents the Background of COMET and Motivation for our work.
Section 3 explains our offloading heuristics. Section 4 describes the methodology used to conduct experiments. Section 5 presents
analysis and results of the applications studied. Section 6 addresses the related work and Section 7 presents our conclusion with
future work.
%-------------------------------------------------------%
%\begin{figure} [thf*]
%\centering
%\begin{tabular}{c}
%\begin{minipage}[b]{0.5\textwidth}
%\includegraphics[width=0.95\textwidth]{figs/router.pdf}
%\end{minipage}
%\end{tabular}
%\caption{Router Architecture.}
%\label{fig:routerarch}
%\end{figure}
%-------------------------------------------------------%
