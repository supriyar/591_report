Recent research has focussed on increasing the computational capabilities of mobile devices. By using code offloading
resource intensive mobile components are identified and offloaded to remote servers in order to be executed
by machines in the cloud. Most of the relevant work in this area have used code annotations and VM synchronization to
implement the client to server offloading. The most noteworthy of these works are MAUI~\cite{maui} and CloneCloud~\cite{ccloud}.

CloneCloud follows an approach that is based on static analysis of the mobile application to partition it.
Based on this partition, the application can be executed in the client or the server. CloneCloud maintains
a clone of the mobile application stack in a virtual machine on the server. This helps is synchronization
when the application is offloaded as the application can be executed at the client or the server seamlessly.
The strategy proposed by MAUI is based on code annotations to determine the methods that can be offloaded to the
cloud. The developer is given the responsibility of generating these annotations in the source code. The MAUI
profiler identifies these annotations at run-time and offloads the methods to the server.

Intelligent offloading decisions has been investigated previously by Flores et.al. ~\cite{fuzzy}. Their approach
generates offloading decisions based on server parameters. By using fuzzy-logic, they improve the offloading decision
over time. This decision engine relies on both, mobile and cloud parameters. Though the ideology of this work is similar
to our intention our approach differs in certain aspects. Our approach differs as we look
at the execution times of methods, using dynamic profiling to determine when to migrate methods to the server and when
to aviod the state transfer and execute locally.

