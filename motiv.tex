In COMET the authors introduced a runtime system that allows migration of threads
between systems depending on their workload. COMET extended distributed
shared memory (DSM) techniques to enable offloading, a process in which
a resource constrained device can execute compute intensive tasks on a
relatively high performance device.

The core idea behind COMET is to manage memory consistency across different
points at a field
granularity using a VM Synchronization primitve. Changes to objects are
tracked at a field level by setting an addtional dirty bit to true if the
field was modified by an end point. The \textit{happens-before} ordering
between reads and writes of different threads helps in merging the changed
fields to achieve a consistent state. A VM synchronization primitive is used
to transfer the states of virtual heaps, stacks, source code (dex files) and
synthetic classes to the other end point.


For every new thread that has to be migrated, a parallel thread is
created on the other end point, if it does not exist already. If a parallel
thread does exist, a VM sync is used to transfer state.
A VM sync is usually initiated when a thread requests for ownership of
an object it doesn't own or when the thread can be migrated to the other
end point for execution. The thread that initiated the sync starts
collecting all the fields that have been changed since last sync. Once the
dirtied fields have been obtained, the stacks that have to be synced are
gathered and all the state changes are then transmitted over to the other
end point. During changed state collection all threads are temporarily suspended
and are resumed just before the data transfer is initiated. The thread that was
migrated waits in a loop waiting for a \textit{resume} message from the other
end point. Communication between parallel threads running on different end points
is via such messages.


To avoid exchanging
the code that the other end point has to resume from, all the source code files are
transferred only on the first syncronization. The current program counter value
gives the next instruction the other end point begins execution from. To support
locking primitives, COMET assigns an owner to each lock (through a flag)
indicating the end point that currently owns the object. When a thread
encounters an object it doesn't own, an ownership transfer request is
initiated to acquire the object, which could result in a VM synchronization.

%TODO: Talk about comet being designed for Java ?
Native functions are methods written in C language that are compiled to run on a
specific architecture and are therefore non-offloadable. These native methods
are invoked through the Java Native Interface and have a faster execution time when
compared to methods written in Java. In COMET the authors identified about 200
native methods that are safe for off-loading and these are manually annotated to indicate
that they can be executed at any end point. The remaining native methods are
however never executed on the remote end point and are migrated back.

Scheduling decision regarding when to offload a thread primarily depends on the following
factors - network conditions and time elapsed since the execution of the last
native method. When a thread encounters instructions that are not to be migrated it
updates the time it encountered the \textit{unsafe point}. Migration decisions are made (from our
understanding of code) periodically. The scheduler decides to migrate a thread if
its execution time since a last unsafe point is higher than a threshold
derived from the network conditions. Network conditions include both bandwidth and the
round trip time. For the first ten syncronizations, the threshold is a weighted
average between the RTT(+variance) and the recent synchronization time, after which
the treshold solely depends on the average of the last 15 syncronization times.

From the implementation of COMET, we came to understand that the RTT is calcualted
only at the beginning of an application's execution. We reasoned that it is infact
correct to only estimate it once as the latency on evaluating it repeatedly for
each scheduling decision would be atleast RTT. We observed that the last sync
time (which is actually the average of the last 15 sync times) captures the
available bandwidth and the network congestion. We also concluded that since
last sync time implicitly captures the latency it was indeed acceptable to
calculate the threshold after a certain interval, based on only the recent sync time.

We began our project with the motivation to further reduce the synchronizations
between the two end points. It is clear that migrating a method that calls a
native method in the future would only result in an abort at the other end point
and then a transfer back to the original end point. With the aid of static analysis
we wanted to predict if a method would call a native method in a fixed future interval
of time. While a naive static analysis approach might immediately seem to be
sufficient for our purpose, it is to be noted that by restricting the migration of a
method based on its future invocation of a native method we would be forcing it to
run on the host end point which could result in a longer execution time to reach the
native method. If the time to execute on server coupled with the sync time
would have been less than the time it took to reach the native method on the client
we would have erred in our decision to not migrate the method.

A second motivation for us began with the observation that upon returning to the client
after an abort at native method on server, the client waits for a duration that is
atleast the recent sync time before it can attempt to offload again. While this
seems to be a reasonable strategy, it appeared the wait didn't take into account
that we returned to the client because of a native method abort.

%TODO: Talk about tcpmux and data transfer - needed because we have to talk about
% recovery mechanisms
