In COMET (Code Offlaod by Migrating Execution Transparently) \cite{comet} 
the authors introduced a runtime system that allows migration of threads 
between systems depending on their workload. COMET extended distributed 
shared memory (DSM) techniques to enable offloading, a process in which
a resource constrained device can execute compute intensive tasks on a 
relatively high performance device. 

The core idea behind COMET is to manage memory consistency across different
points at a field 
granularity using a VM Synchronization primitve. Changes to objects are 
tracked at a field level by setting an addtional dirty bit to true if the 
field was modified by an end point. The \textit{happens-before} ordering 
between reads and writes of different threads helps in merging the changed 
fields to achieve a consistent state. A VM synchronization primitive is used 
to transfer the states of virtual heaps, stacks, source code (dex files) and 
synthetic classes to the other end point. 


A VM sync is usually initiated when a thread requests for ownership of 
an object it doesn't own or when the thread can be migrated to the other 
end point for execution. The thread that initiated the sync starts 
collecting all the fields that have been changed since last sync. Once the 
dirtied fields have been obtained, the stacks that have to be synced are 
gathered and all the state changes are then transmitted over to the other 
end point. During changed state collection all threads are temporarily suspended
and are resumed just before the data transfer is initiated. To avoid exchanging
the code that the other end point has to resume from, all the source code files are 
transferred only on the first syncronization. The current program counter value 
gives the next instruction the other end point begins execution from. To support 
locking primitives, COMET assigns an owner to each lock (through a flag) 
indicating the end point that currently owns the object. When a thread 
encounters an object it doesn't own, an ownership transfer request is 
initiated to acquire the object, which could result in a VM synchronization. 


