

\ignore{
\subsection{Network Configurations Studied}

We evaluate the network performance for four different
switch allocation schemes - \textbf{Separable input-first (IF)}, \textbf{Wavefront
(WF)}, \textbf{Augmented Path (AP)} and \textbf{VIX}. Unless otherwise
stated, all VIX configurations have two virtual inputs per input
port. All our experiments are performed
for 64 node networks.
 For each of these
configurations, we study the average packet latency and average
network throughput for each of the four switch allocation
techniques. To study the allocation techniques independent of their
implementation limitations, we assume equal cycle time for all switch
allocation techniques. The packet latency is reported in $cycles$ and
network throughput in $packets/cycle/node$ or $flits/cycle$.
The routers have a 128 bit datapath width, and support 6 virtual
channels per input port. We evaluate the different network configurations with uniform random
  statistical traffic with a packet size of 512 bits (i.e.~4
  flits).
%-------------------------------------------------------%
\begin{figure*} [tbf*]
\centering
\begin{tabular}{cc}
\begin{minipage}[b]{0.5\textwidth}
\includegraphics[height=0.2\textheight]{figs/mesh-thpt.pdf}
\center
\vspace{-4mm}
(a)
\end{minipage}
&
\begin{minipage}[b]{0.5\textwidth}
\includegraphics[height=0.2\textheight]{figs/mesh-lat.pdf}
\center
\vspace{-4mm}
(b)
\end{minipage}
\end{tabular}
\caption{Network throughput and packet latency for a mesh topology.}
\label{fig:mesh}
\end{figure*}
%-------------------------------------------------------%

%-------------------------------------------------------%
\begin{figure*} [tbf*]
\centering
\begin{tabular}{cc}
\begin{minipage}[b]{0.5\textwidth}
\includegraphics[height=0.2\textheight]{figs/fbfly-thpt.pdf}
\center
\vspace{-4mm}
(a)
\end{minipage}
&
\begin{minipage}[b]{0.5\textwidth}
\includegraphics[height=0.2\textheight]{figs/fbfly-lat.pdf}
\center
\vspace{-4mm}
(b)
\end{minipage}
\end{tabular}
\caption{Network throughput and packet latency for a Flattened Butterfly topology.}
\label{fig:fbfly}
%\vspace{-2mm}
\end{figure*}
%-------------------------------------------------------%
}
